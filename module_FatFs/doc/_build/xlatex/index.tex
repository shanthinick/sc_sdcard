% Generated by Sphinx.
\def\sphinxdocclass[english]{xmosmodern}
\documentclass[  collection]{xmosmodern}
\usepackage[utf8]{inputenc}
\DeclareUnicodeCharacter{00A0}{\nobreakspace}

\usepackage{longtable}



\title{FAT Filesystem component}
\date{December 16, 2013}
\author{}
\newcommand{\sphinxlogo}{}
\newcommand{\releasename}{Release}
\usepackage{xsphinx}
\usepackage{threeparttable}
\usepackage{fancyvrb}
\usepackage{indent}
\renewcommand\bfcode\textbf
\renewcommand\bf\textbf
\graphicspath{{./}{./images/}}
\makeindex

\newcommand\PYGZat{@}
\newcommand\PYGZlb{[}
\newcommand\PYGZrb{]}

\setlength{\emergencystretch}{8em}
\start

\maketitle
\pretoc
\phantomsection\label{index::doc}

%summary!

% NON-FULLWIDTH SECTION



% NON-FULLWIDTH SECTION
\clearpage
\chapter{Overview}
\label{overview:fat-filesystem-component}\label{overview::doc}\label{overview:overview}%summary!
\begin{inthisdocument}
\item \nameref{overview:features}
\item \nameref{overview:memory-requirements}
\item \nameref{overview:resource-requirements}
\item \nameref{overview:performance}
\end{inthisdocument}



The FAT Filesystem module is used to read/write files into the FAT filesystem



% NON-FULLWIDTH SECTION
\section{Features}
\label{overview:features}\begin{itemize}
\item   APIs to access FAT filesystem

\item   Port of FatFS - FAT file system module R0.09 (C)ChaN, 2011 (\xurl{http://elm-chan.org/fsw/ff/00index\_e.html}).

\end{itemize}




% NON-FULLWIDTH SECTION
\section{Memory requirements}
\label{overview:memory-requirements}
\begin{tabular}{ll}
\Toprule
\textbf{Resource} & \textbf{Usage}\\
\midrule
Stack & 480 bytes\\
Program & 15 Kbytes\\
\bottomrule
\end{tabular}




% NON-FULLWIDTH SECTION
\section{Resource requirements}
\label{overview:resource-requirements}
\begin{tabular}{ll}
\Toprule
\textbf{Resource} & \textbf{Usage}\\
\midrule
Timers & 1\\
Clocks & 1\\
Threads & 1\\
\bottomrule
\end{tabular}




% NON-FULLWIDTH SECTION
\section{Performance}
\label{overview:performance}

SPI mode 3 is used by the SD card driver. The performance measured includes FAT filesystem performance along with the SD card driver on SPI interface.

\begin{tabular}{ll}
\Toprule
\textbf{R/W} & \textbf{PERFORMANCE}\\
\midrule
WRITE & 1061 KBytes/s\\
READ & 314 KBytes/s\\
\bottomrule
\end{tabular}




% NON-FULLWIDTH SECTION
\clearpage
\chapter{Hardware requirements}
\label{hw::doc}\label{hw:evaluation-platforms}%summary!
\begin{inthisdocument}
\item \nameref{hw:sec-hardware-platforms}
\item \nameref{hw:demonstration-applications}
\end{inthisdocument}




% NON-FULLWIDTH SECTION
\section{Recommended hardware}
\label{hw:sec-hardware-platforms}\label{hw:recommended-hardware}


% NON-FULLWIDTH SECTION
\subsection{sliceKIT}
\label{hw:slicekit}

This module may be evaluated using the sliceKIT modular development platform, available from digikey. Required board SKUs are:
\begin{itemize}
\item   XP-SKC-L2 (sliceKIT L2 Core Board)

\item   XA-SK-XTAG2 (sliceKIT xTAG adaptor)

\item   XA-SK-FLASH 1V0 Slice Card

\end{itemize}




% NON-FULLWIDTH SECTION
\section{Demonstration applications}
\label{hw:demonstration-applications}


% NON-FULLWIDTH SECTION
\subsection{Display controller application}
\label{hw:display-controller-application}\begin{itemize}
\item   Package: sc\_sdcard

\item   Application: app\_sdcard\_test

\end{itemize}



This demo uses the \verb`module_FatFs` along with the \verb`module_sdcardSPI` and \verb`module_spi_master`.


Required board SKUs for this demo are:
\begin{itemize}
\item   XP-SKC-L16 (sliceKIT L16 Core Board) plus XA-SK-XTAG2 (sliceKIT xTAG adaptor)

\end{itemize}




% FULLWIDTH SECTION (with summary)
\clearpage
\chapter{API}
\label{api:sec-fatfs-api}\label{api::doc}\label{api:project-structure}%summary!
\begin{inthisdocument}
\item \nameref{api:configuration-defines}
\item \nameref{api:api}
\end{inthisdocument}

\begin{fullwidth} % chapter!
\begin{description}
\item[To build a project including the \ttfamily module\_FatFs the following modules are required:]\begin{itemize}
\item   module: module\_sdcardSPI module\_spi\_master

\end{itemize}


\end{description}



The below section details the APIs in the application. For details about the FatFS APIs please refer to the respective repositories.


\section{Configuration defines}
\label{api:configuration-defines}

The \verb`module_FatFs` requires configurations defined in ffconf.h. The module requires nothing to be additionally defined.
However defines can be tuned in ffconf.h. Some of the defines are:
\begin{description}
\item[\textbf{\_FS\_TINY}]

This defines if the sector buffer in the file system object or the individual file object should be used. When \_FS\_TINY is set to 1, FatFs uses the sector buffer in the file system object.

\item[\textbf{\_FS\_READONLY}]

Setting \_FS\_READONLY to 1 defines read only configuration. By default this is set to 0.

\item[\textbf{\_FS\_MINIMIZE}]

This define sets minimization level to remove some functions. By default this is set to 0 to include full set of functions.

\item[\textbf{\_FS\_READONLY}]

Setting \_FS\_READONLY to 1 defines read only configuration. By default this is set to 0.

\end{description}



\section{API}
\label{api:api}\begin{description}
\item[The{}`{}`module\_FatFs{}`{}` functionality is defined in]\begin{itemize}
\item   \verb`ff.c`

\item   \verb`ccsbcs.c_`

\item   \verb`ff.h`

\item   \verb`ffconf.h`

\item   \verb`integer.h`

\item   \verb`diskio.h`

\end{itemize}


\end{description}



The FatFs module provides APIs to read/write files to SD card.


The FatFs APIs are as follows:
\index{f\_mount (C function)}

\texttt{\phantomsection\label{api:f_mount}FRESULT f\_mount(BYTE vol, FATFS *fs)}

\vspace{-2mm}
\sloppy
\begin{indentation*}{\blockindentlen}{0mm}

Function to mount/unmount file system object to the FatFs module.


This function has the following parameters:

\begin{option}

\item[vol]Logical drive number to be mounted/unmounted n.
\item[fs]Pointer to new file system object (NULL for unmount)
\end{option}
\vspace{-3mm}


\end{indentation*}
\fussy
\index{f\_open (C function)}

\texttt{\phantomsection\label{api:f_open}FRESULT f\_open(FIL *fp, const TCHAR *path, BYTE mode)}

\vspace{-2mm}
\sloppy
\begin{indentation*}{\blockindentlen}{0mm}

Function to open or create a file.


This function has the following parameters:

\begin{option}

\item[fp]Pointer to the blank file object.
\item[path]Pointer to the file name.
\item[mode]Access mode and file open mode flags.
\end{option}
\vspace{-3mm}


\end{indentation*}
\fussy
\index{f\_read (C function)}

\texttt{\phantomsection\label{api:f_read}FRESULT f\_read(FIL *fp, void *buff, UINT btr, UINT *br)}

\vspace{-2mm}
\sloppy
\begin{indentation*}{\blockindentlen}{0mm}

Function to read data from a file.


This function has the following parameters:

\begin{option}

\item[fp]Pointer to the file object.
\item[buff]Pointer to data buffer.
\item[btr]Number of bytes to read.
\item[br]Pointer to number of bytes read.
\end{option}
\vspace{-3mm}


\end{indentation*}
\fussy
\index{f\_lseek (C function)}

\texttt{\phantomsection\label{api:f_lseek}FRESULT f\_lseek(FIL *fp, DWORD ofs)}

\vspace{-2mm}
\sloppy
\begin{indentation*}{\blockindentlen}{0mm}

Function to move file pointer of a file object.


This function has the following parameters:

\begin{option}

\item[fp]Pointer to the file object .
\item[ofs]File pointer from top of file .
\end{option}
\vspace{-3mm}


\end{indentation*}
\fussy
\index{f\_close (C function)}

\texttt{\phantomsection\label{api:f_close}FRESULT f\_close(FIL *fp)}

\vspace{-2mm}
\sloppy
\begin{indentation*}{\blockindentlen}{0mm}

Function to close an open file object.


This function has the following parameters:

\begin{option}

\item[fp]Pointer to the file object to be closed
\end{option}
\vspace{-3mm}


\end{indentation*}
\fussy
\index{f\_opendir (C function)}

\texttt{\phantomsection\label{api:f_opendir}FRESULT f\_opendir(DIR *dj, const TCHAR *path)}

\vspace{-2mm}
\sloppy
\begin{indentation*}{\blockindentlen}{0mm}

Function to create a directory object.


This function has the following parameters:

\begin{option}

\item[dj]Pointer to directory object to create.
\item[path]Pointer to the directory path.
\end{option}
\vspace{-3mm}


\end{indentation*}
\fussy
\index{f\_readdir (C function)}

\texttt{\phantomsection\label{api:f_readdir}FRESULT f\_readdir(DIR *dj, FILINFO *fno)}

\vspace{-2mm}
\sloppy
\begin{indentation*}{\blockindentlen}{0mm}

Function to read directory entry in sequence.


This function has the following parameters:

\begin{option}

\item[dj]Pointer to the open directory object.
\item[fno]Pointer to file information to return.
\end{option}
\vspace{-3mm}


\end{indentation*}
\fussy
\index{f\_stat (C function)}

\texttt{\phantomsection\label{api:f_stat}FRESULT f\_stat(const TCHAR *path, FILINFO *fno)}

\vspace{-2mm}
\sloppy
\begin{indentation*}{\blockindentlen}{0mm}

Function to get the file status.


This function has the following parameters:

\begin{option}

\item[path]Pointer to the file path.
\item[fno]Pointer to file information to return .
\end{option}
\vspace{-3mm}


\end{indentation*}
\fussy
\index{f\_write (C function)}

\texttt{\phantomsection\label{api:f_write}FRESULT f\_write(FIL *fp, const void *buff, UINT btw, UINT *bw)}

\vspace{-2mm}
\sloppy
\begin{indentation*}{\blockindentlen}{0mm}

Function to write data to the file.


This function has the following parameters:

\begin{option}

\item[fp]Pointer to the file object.
\item[buff]Pointer to the data to be written.
\item[btw]Number of bytes to write.
\item[bw]Pointer to number of bytes written.
\end{option}
\vspace{-3mm}


\end{indentation*}
\fussy
\index{f\_getfree (C function)}

\texttt{\phantomsection\label{api:f_getfree}FRESULT f\_getfree(const TCHAR *path, DWORD *nclst, FATFS **fatfs)}

\vspace{-2mm}
\sloppy
\begin{indentation*}{\blockindentlen}{0mm}

Function to get number of free clusters on the drive.


This function has the following parameters:

\begin{option}

\item[path]Pointer to the logical drive number (root dir).
\item[nclst]Pointer to the variable to return number of free clusters.
\item[fatfs]Pointer to pointer to corresponding file system object to return.
\end{option}
\vspace{-3mm}


\end{indentation*}
\fussy
\index{f\_truncate (C function)}

\texttt{\phantomsection\label{api:f_truncate}FRESULT f\_truncate(FIL *fp)}

\vspace{-2mm}
\sloppy
\begin{indentation*}{\blockindentlen}{0mm}

Function to truncate a file.


This function has the following parameters:

\begin{option}

\item[fp]Pointer to the file object to be truncated.
\end{option}
\vspace{-3mm}


\end{indentation*}
\fussy
\index{f\_unlink (C function)}

\texttt{\phantomsection\label{api:f_unlink}FRESULT f\_unlink(const TCHAR *path)}

\vspace{-2mm}
\sloppy
\begin{indentation*}{\blockindentlen}{0mm}

Function to delete an existing file or directory.


This function has the following parameters:

\begin{option}

\item[path]Pointer to the file or directory path.
\end{option}
\vspace{-3mm}


\end{indentation*}
\fussy
\index{f\_mkdir (C function)}

\texttt{\phantomsection\label{api:f_mkdir}FRESULT f\_mkdir(const TCHAR *path)}

\vspace{-2mm}
\sloppy
\begin{indentation*}{\blockindentlen}{0mm}

Function to create a new directory.


This function has the following parameters:

\begin{option}

\item[path]Pointer to the directory path.
\end{option}
\vspace{-3mm}


\end{indentation*}
\fussy
\index{f\_chmod (C function)}

\texttt{\phantomsection\label{api:f_chmod}FRESULT f\_chmod(const TCHAR *path, BYTE value, BYTE mask)}

\vspace{-2mm}
\sloppy
\begin{indentation*}{\blockindentlen}{0mm}

Function to change attribute of a file or directory.


This function has the following parameters:

\begin{option}

\item[path]Pointer to the file path.
\item[value]Attribute bits.
\item[mask]Attribute mask to change.
\end{option}
\vspace{-3mm}


\end{indentation*}
\fussy
\index{f\_rename (C function)}

\texttt{\phantomsection\label{api:f_rename}FRESULT f\_rename(const TCHAR *path\_old, const TCHAR *path\_new)}

\vspace{-2mm}
\sloppy
\begin{indentation*}{\blockindentlen}{0mm}

Function to rename a file or directory.


This function has the following parameters:

\begin{option}

\item[path\_old]Pointer to the old name.
\item[path\_new]Pointer to the new name.
\end{option}
\vspace{-3mm}


\end{indentation*}
\fussy
\index{f\_chdrive (C function)}

\texttt{\phantomsection\label{api:f_chdrive}FRESULT f\_chdrive(BYTE drv)}

\vspace{-2mm}
\sloppy
\begin{indentation*}{\blockindentlen}{0mm}

Function to change current drive.


This function has the following parameters:

\begin{option}

\item[drv]Function to change current drive.
\end{option}
\vspace{-3mm}


\end{indentation*}
\fussy
\index{f\_chdir (C function)}

\texttt{\phantomsection\label{api:f_chdir}FRESULT f\_chdir(const TCHAR *path)}

\vspace{-2mm}
\sloppy
\begin{indentation*}{\blockindentlen}{0mm}

Function to change current directory.


This function has the following parameters:

\begin{option}

\item[path]Pointer to the directory path.
\end{option}
\vspace{-3mm}


\end{indentation*}
\fussy
\index{f\_getcwd (C function)}

\texttt{\phantomsection\label{api:f_getcwd}FRESULT f\_getcwd(TCHAR *path, UINT sz\_path)}

\vspace{-2mm}
\sloppy
\begin{indentation*}{\blockindentlen}{0mm}

Function to get current directory.


This function has the following parameters:

\begin{option}

\item[path]Pointer to the directory path.
\item[sz\_path]Size of path.
\end{option}
\vspace{-3mm}


\end{indentation*}
\fussy
\index{f\_fdisk (C function)}

\texttt{\phantomsection\label{api:f_fdisk}FRESULT f\_fdisk(BYTE pdrv, const DWORD szt{[}{]}, void *work)}

\vspace{-2mm}
\sloppy
\begin{indentation*}{\blockindentlen}{0mm}

Function to divide a physical drive into multipe partitions.


This function has the following parameters:

\begin{option}

\item[pdrv]Physical drive number.
\item[szt]Pointer to the size table for each partitions.
\item[work]Pointer to the working buffer.
\end{option}
\vspace{-3mm}


\end{indentation*}
\fussy
\index{f\_putc (C function)}

\texttt{\phantomsection\label{api:f_putc}int f\_putc(TCHAR c, FIL *fp)}

\vspace{-2mm}
\sloppy
\begin{indentation*}{\blockindentlen}{0mm}

Function to put a character to the file.


This function has the following parameters:

\begin{option}

\item[c]The character to be output.
\item[fp]Pointer to the file object.
\end{option}
\vspace{-3mm}


\end{indentation*}
\fussy
\index{f\_printf (C function)}

\texttt{\phantomsection\label{api:f_printf}int f\_printf(FIL *fil, const TCHAR *str, ...)}

\vspace{-2mm}
\sloppy
\begin{indentation*}{\blockindentlen}{0mm}

Function to print a formatted string into a file.


This function has the following parameters:

\begin{option}

\item[fil]Pointer to the file object.
\item[str]Pointer to the formatted string.
\item[...]Optional arguments...
\end{option}
\vspace{-3mm}


\end{indentation*}
\fussy


The FatFs APIs use the module\_sdcardSPI APIs.
\end{fullwidth}%


% NON-FULLWIDTH SECTION
\clearpage
\chapter{Programming guide}
\label{programming:}\label{programming:programming-guide}\label{programming::doc}%last summary



\posttoc

\finish
