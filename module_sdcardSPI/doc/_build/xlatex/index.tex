% Generated by Sphinx.
\def\sphinxdocclass[english]{xmosmodern}
\documentclass[  collection]{xmosmodern}
\usepackage[utf8]{inputenc}
\DeclareUnicodeCharacter{00A0}{\nobreakspace}

\usepackage{longtable}



\title{SD card component}
\date{December 04, 2013}
\author{}
\newcommand{\sphinxlogo}{}
\newcommand{\releasename}{Release}
\usepackage{xsphinx}
\usepackage{threeparttable}
\usepackage{fancyvrb}
\usepackage{indent}
\renewcommand\bfcode\textbf
\renewcommand\bf\textbf
\graphicspath{{./}{./images/}}
\makeindex

\newcommand\PYGZat{@}
\newcommand\PYGZlb{[}
\newcommand\PYGZrb{]}

\setlength{\emergencystretch}{8em}
\start

\maketitle
\pretoc
\phantomsection\label{index::doc}

%summary!

% NON-FULLWIDTH SECTION



% NON-FULLWIDTH SECTION
\clearpage
\chapter{Overview}
\label{overview:overview}\label{overview:sd-card-component}\label{overview::doc}%summary!
\begin{inthisdocument}
\item \nameref{overview:features}
\item \nameref{overview:memory-requirements}
\item \nameref{overview:resource-requirements}
\item \nameref{overview:performance}
\end{inthisdocument}



The SD card module is used to talk to the SD card through the SPI interface.



% NON-FULLWIDTH SECTION
\section{Features}
\label{overview:features}\begin{itemize}
\item   Low level functions for accessing SD card

\item   Uses SPI interface

\end{itemize}




% NON-FULLWIDTH SECTION
\section{Memory requirements}
\label{overview:memory-requirements}
\begin{tabular}{ll}
\Toprule
\textbf{Resource} & \textbf{Usage}\\
\midrule
Stack & 220 bytes\\
Program & 5 KB\\
\bottomrule
\end{tabular}




% NON-FULLWIDTH SECTION
\section{Resource requirements}
\label{overview:resource-requirements}
\begin{tabular}{ll}
\Toprule
\textbf{Resource} & \textbf{Usage}\\
\midrule
Clocks & 1\\
Threads & 1\\
\bottomrule
\end{tabular}




% NON-FULLWIDTH SECTION
\section{Performance}
\label{overview:performance}

The achievable effective bandwidth varies according to the available xCORE MIPS. The maximum pixel clock supported is 25MHz.



% NON-FULLWIDTH SECTION
\clearpage
\chapter{Hardware requirements}
\label{hw::doc}\label{hw:evaluation-platforms}%summary!
\begin{inthisdocument}
\item \nameref{hw:sec-hardware-platforms}
\item \nameref{hw:demonstration-applications}
\end{inthisdocument}




% NON-FULLWIDTH SECTION
\section{Recommended hardware}
\label{hw:sec-hardware-platforms}\label{hw:recommended-hardware}


% NON-FULLWIDTH SECTION
\subsection{sliceKIT}
\label{hw:slicekit}

This module may be evaluated using the sliceKIT modular development platform, available from digikey. Required board SKUs are:
\begin{itemize}
\item   XP-SKC-L2 (sliceKIT L2 Core Board)

\item   XA-SK-XTAG2 (sliceKIT xTAG adaptor)

\item   XA-SK-FLASH 1V0 Slice Card

\end{itemize}




% NON-FULLWIDTH SECTION
\section{Demonstration applications}
\label{hw:demonstration-applications}


% NON-FULLWIDTH SECTION
\subsection{Display controller application}
\label{hw:display-controller-application}\begin{itemize}
\item   Package: sc\_sdcard

\item   Application: app\_sdcard\_test

\end{itemize}



This demo uses the \verb`module_FatFs` along with the \verb`module_sdcardSPI`, \verb`module_spi_master`.


Required board SKUs for this demo are:
\begin{itemize}
\item   XP-SKC-L16 (sliceKIT L16 Core Board) plus XA-SK-XTAG2 (sliceKIT xTAG adaptor)

\end{itemize}




% FULLWIDTH SECTION (with summary)
\clearpage
\chapter{API}
\label{api::doc}\label{api:sec-sdcardspi-api}\label{api:project-structure}%summary!
\begin{inthisdocument}
\item \nameref{api:configuration-defines}
\item \nameref{api:api}
\end{inthisdocument}

\begin{fullwidth} % chapter!
\begin{description}
\item[To build a project including the \ttfamily module\_sdcardSPI the following module is required:]\begin{itemize}
\item   module: module\_spi\_master

\end{itemize}


\end{description}



The below section details the APIs in the SD card module. For details about the FatFS APIs please refer to the respective repositories.


\section{Configuration defines}
\label{api:configuration-defines}

The \verb`module_sdcardSPI` requires a configuration defined in spi\_conf.h. The module requires nothing to be additionally defined.
\begin{description}
\item[\textbf{SPI\_MASTER\_SD\_CARD\_COMPAT}]

This defines the SPI modifications needed for SD card usage. This needs to be set to 1 for SD cards to use SPI interface.

\end{description}



\section{API}
\label{api:api}\begin{description}
\item[The{}`{}`module\_sdcardSPI{}`{}` functionality is defined in]\begin{itemize}
\item   \verb`SDCardHostSPI.xc`

\item   \verb`spi_conf.h`

\end{itemize}


\end{description}



The SD card module provides APIs to read/write data to SD card.


The SD card APIs are as follows:
\index{disk\_initialize (C function)}

\texttt{\phantomsection\label{api:disk_initialize}DSTATUS disk\_initialize(BYTE drv)}

\vspace{-2mm}
\sloppy
\begin{indentation*}{\blockindentlen}{0mm}

Function to initialize disk.


This function has the following parameters:

\begin{option}

\item[drv]Physical drive nmuber (0).
\end{option}
\vspace{-3mm}


\end{indentation*}
\fussy
\index{disk\_status (C function)}

\texttt{\phantomsection\label{api:disk_status}DSTATUS disk\_status(BYTE drv)}

\vspace{-2mm}
\sloppy
\begin{indentation*}{\blockindentlen}{0mm}

Function to open or create a file.


This function has the following parameters:

\begin{option}

\item[drv]Drive number (always 0).
\end{option}
\vspace{-3mm}


\end{indentation*}
\fussy
\index{disk\_read (C function)}

\texttt{\phantomsection\label{api:disk_read}DRESULT disk\_read(BYTE drv, BYTE buff{[}{]}, DWORD sector, BYTE count)}

\vspace{-2mm}
\sloppy
\begin{indentation*}{\blockindentlen}{0mm}

Function to read data from a disk.


This function has the following parameters:

\begin{option}

\item[drv]Physical drive nmuber (0).
\item[buff]Pointer to the data buffer to store read data.
\item[sector]Start sector number (LBA).
\item[count]Sector count (1..128).
\end{option}
\vspace{-3mm}


\end{indentation*}
\fussy
\index{disk\_write (C function)}

\texttt{\phantomsection\label{api:disk_write}DRESULT disk\_write(BYTE drv, const BYTE buff{[}{]}, DWORD sector, BYTE count)}

\vspace{-2mm}
\sloppy
\begin{indentation*}{\blockindentlen}{0mm}

Function to write data to the disk.


This function has the following parameters:

\begin{option}

\item[drv]Physical drive nmuber (0).
\item[buff]Pointer to the data to be written.
\item[sector]Start sector number (LBA).
\item[count]Sector count (1..128).
\end{option}
\vspace{-3mm}


\end{indentation*}
\fussy
\index{disk\_ioctl (C function)}

\texttt{\phantomsection\label{api:disk_ioctl}DRESULT disk\_ioctl(BYTE drv, BYTE ctrl, BYTE buff{[}{]})}

\vspace{-2mm}
\sloppy
\begin{indentation*}{\blockindentlen}{0mm}

Function to control device specific features and miscellaneous functions.


This function has the following parameters:

\begin{option}

\item[drv]Physical drive nmuber (0).
\item[ctrl]Control code.
\item[buff]Buffer to send/receive control data.
\end{option}
\vspace{-3mm}


\end{indentation*}
\fussy


The SD card APIs use the module\_spi\_master APIs.
\end{fullwidth}%


% NON-FULLWIDTH SECTION
\clearpage
\chapter{Programming guide}
\label{programming:}\label{programming:programming-guide}\label{programming::doc}%summary!
\begin{inthisdocument}
\item \nameref{programming:shared-memory-interface}
\item \nameref{programming:source-code-structure}
\item \nameref{programming:executing-the-project}
\item \nameref{programming:software-requirements}
\end{inthisdocument}




% NON-FULLWIDTH SECTION
\section{Shared memory interface}
\label{programming:shared-memory-interface}

The display controller uses a shared memory interface to move the large amount of data around from tile to tile efficiently. This means that the \verb`display_controller`, \verb`sdram_server` and \verb`lcd_server` must be one the same tile.



% NON-FULLWIDTH SECTION
\section{Source code structure}
\label{programming:source-code-structure}
\begin{figure}[h]\begin{sidecaption}{Project structure}
\small
\begin{tabularx}{\linewidth}{llY}
\Toprule
\textbf{Project} & \textbf{File} & \textbf{Description}\\
\midrule
module\_display\_controller & \verb`display_controller.h` & Header file containing the APIs for the display controller component.\\
 & \verb`display_controller.xc` & File containing the implementation of the display controller component.\\
 & \verb`display_controller_client.xc` & File containing the implementation of the display controller client functions.\\
 & \verb`display_controller_internal.h` & Header file containing the user configurable defines for the display controller component.\\
 & \verb`transitions.h` & Header file containing the APIs for the display controller transitions.\\
 & \verb`transitions.xc` & File containing the implementation of the display controller transitions.\\
\bottomrule
\end{tabularx}

\end{sidecaption}
\end{figure} \DocumentFooterFix



% NON-FULLWIDTH SECTION
\section{Executing the project}
\label{programming:executing-the-project}

The module by itself cannot be built or executed separately - it must be linked in to an application. Once the module is linked to the application, the application can be built and tested for driving a LCD screen.
\begin{description}
\item[The following modules should be added to the list of MODULES in order to link the component to the application project]\begin{enumerate}
\item   \verb`module_display_controller`

\item   \verb`module_lcd`

\item   \verb`module_sdram`

\end{enumerate}


\item[Now the module is linked to the application and can be used directly. Additionally, if the use of the touch controller is nessessary then]\begin{enumerate}
\item   \verb`module_touch_controller_lib` or \verb`module_touch_controller_server`

\item   \verb`module_i2c_master`

\end{enumerate}


\end{description}



should be added to the list of MODULES.



% NON-FULLWIDTH SECTION
\section{Software requirements}
\label{programming:software-requirements}

The module is built on xTIMEcomposer version 12.0
The module can be used in version 12.0 or any higher version of xTIMEcomposer.



% NON-FULLWIDTH SECTION
\clearpage
\chapter{Example applications}
\label{examples:example-application}\label{examples::doc}%last summary
\begin{inthisdocument}
\item \nameref{examples:app-sdcard-test}
\item \nameref{examples:application-notes}
\end{inthisdocument}



This tutorial describes a demo application that reads and writes files on SD card using the SPI interface.



% NON-FULLWIDTH SECTION
\section{app\_sdcard\_test}
\label{examples:app-sdcard-test}\begin{itemize}
\item   Demonstrate read and write files on SD card using the SPI interface. At the end, it prints the read/write performances of the filesystem on SD card using the SPI.

\end{itemize}




% NON-FULLWIDTH SECTION
\section{Application notes}
\label{examples:application-notes}


% NON-FULLWIDTH SECTION
\subsection{Getting started}
\label{examples:getting-started}\begin{enumerate}
\item   Connect XA-SK-FLASH 1V0 Slice Card to the XP-SKC-L16 sliceKIT Core board using the connector marked with the \verb`TRIANGLE`.

\item   Connect the xTAG Adapter to sliceKIT Core board, and connect xTAG-2 to the adapter.

\item   Connect the xTAG-2 to host PC. Note that the USB cable is not provided with the sliceKIT starter kit.

\item   Set the \verb`XMOS LINK` to \verb`OFF` on the xTAG Adapter(XA-SK-XTAG2).

\item   Make sure the SD card slot in XA-SK-FLASH slice has a Class-4 SD card in it.

\item   Switch on the power supply to the sliceKIT Core board.

\end{enumerate}




% NON-FULLWIDTH SECTION
\subsection{Import and Build the Application}
\label{examples:import-and-build-the-application}\begin{enumerate}
\item   Open xTIMEcomposer and check that it is operating in online mode. Open the edit perspective (Window-\textgreater{}Open Perspective-\textgreater{}XMOS Edit).

\item   Locate the \verb`'SD card demo'` item in the xSOFTip pane on the bottom left of the window and drag it into the Project Explorer window in the xTIMEcomposer. This will also cause the modules on which this application depends to be imported as well.

\item   Click on the app\_sdcard\_test item in the Explorer pane then click on the build icon (hammer) in xTIMEcomposer. Check the console window to verify that the application has built successfully.

\end{enumerate}




% NON-FULLWIDTH SECTION
\subsection{Run the Application}
\label{examples:run-the-application}

Now that the application has been compiled, the next step is to run it on the sliceKIT Core Board using the tools to load the application over JTAG (via the xTAG-2 and xTAG Adapter card) into the xCORE multicore microcontroller.
\begin{enumerate}
\item   Select the file \verb`app_sdcard_test.xe` in the \verb`app_display_controller_demo` project from the Project Explorer.

\item   Click on the \verb`Run` icon (the white arrow in the green circle).

\item   At the \verb`Select Device` dialog select \verb`XMOS xTAG-2 connect to L1[0..1]` and click \verb`OK`.

\item   The application starts executing and reads/writes contents into SD card.

\end{enumerate}




\posttoc

\finish
